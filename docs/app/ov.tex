%%%%%%%%%%%%%%%%%%%%%%%%%%%%%%%%%%%%%%%%%%%%%%%%%%%%%%%%%%%%%%%%%%%%%%%%%%%%%%%
\section{Simulation Model Overview}\label{mod.ov}
The simulation model is a discrete-time, stochastic individual-based model
representing an open population
of approximately $N$ sexual and gender minority (SGM) individuals aged
\note{10--59}
     {From some reading, it seems school is a major context where SGM violence occurs,
      and this can start before SGM are "out",
      so I changed the minimum age for modelling events to 10 from 15.}
including men who have sex with men (MSM) and transgender women (TGW) in
\note{an illustrative Sub-Saharan African context.}{To discuss eventually ...}
The model aims to represent the sexual life-course events of SGM with respect to
the potential causal pathways in Figure~\ref{fig:dag},
with the approximate representation shown in Figure~\ref{fig:model}:
\begin{enumerate}[label=\roman*)]
  \item exposure to SGM violence
  \item distal mediators, including:
        depression and hazardous drinking
  \item proximal determinants of HIV risk, including:
        sexual partnership dynamics and condom use
\end{enumerate}
SGM violence is conceptualized as
\note{any verbal and/or physical violence}
     {I know we may want to split up verbal and physical violence later,
      but leaving as combined for now ...}
on the basis of an individuals' gender and/or sexual identity.
\par
Individuals enter the model in a ``null state'', reflecting
no prior SGM violence, depression, \hazdrink, or sexual activity.
At each timestep, individuals in the model can then experience
any combination of the following events:
exposure to SGM violence,
depression onset or recovery,
hazardous drinking onset or recovery,
non-heterosexual partnership formation or dissolution,
and sex within modelled partnerships (including whether a condom is used).
Rates of ``downstream'' events are influenced by
the current and possibly past states of individuals' ``upstream'' variables,
as described below in \sref{mod.par.evt}.
\begin{figure}[b]
  \centering\includegraphics[scale=1]{dag}
  \caption{Directed graph of the modelled causal pathways}
  \label{fig:dag}
  \floatfoot{
    \emph{Colours} denote:
    green: structural factor;
    blue: distal mediators;
    purple: proximal determinants;
    red: outcome.
    \emph{Lines} denote that the modelled effect depends on:
    solid: current state of upstream variable;
    dashed: current and past state of upstream variable.
    \emph{RR} denote relative rates
    (see Table~\ref{tab:par} for definitions).
    Not shown: effect of age on all variables.}
\end{figure}
\begin{figure}
  \centering\includegraphics[scale=1]{model}
  \caption{Conceptual diagram of states related to
    SGM violence, depression, and \hazdrink}
  \floatfoot{
    \emph{Colours} denote approximate HIV risk
    (mediated by sexual partnership turnover and condom use)
    where blue: lowest HIV risk, red: highest HIV risk.
    \emph{Arrows} denote possible state transitions, where
    depression and \hazdrink transitions are shown for first layer only,
    and SGM violence transitions are applied to all compartment stacks.}
  \label{fig:model}
\end{figure}
\begin{table}
  \caption{Notation reference}
  \label{tab:notation}
  \centering\begin{tabular}{ll}
  \toprule
  Symbol & Meaning \\
  \midrule
  $i$    & \emph{index:} individual \\
  $k$    & \emph{index:} partnership \\
  $z$    & \emph{index:} timestep \\
  $t$    & \emph{index:} time (days) \\
  $\dt$  & \emph{duration:} one timestep \\
  \midrule
  $a$    & \emph{status:} individual's age \\
  $v$    & \emph{event:} SGM violence exposure \\
  $d$    & \emph{event:} depression onset / \emph{status:} current depression \\
  $d'$   & \emph{status:} any previous depression \\
  $\d$   & \emph{event:} depression recovery \\
  $h$    & \emph{event:} \hazdrink onset / \emph{status:} current \hazdrink \\
  $h'$   & \emph{status:} any previous \hazdrink \\
  $\h$   & \emph{event:} \hazdrink recovery \\
  $\du$  & \emph{duration:} time since violence exposure or depression / \hazdrink onset \\
  $p$    & \emph{event:} sexual partnership formation \\
  $\p[]$ & \emph{event:} sexual partnership dissolution \\
  $c$    & \emph{event:} condom use \\
  \midrule
  $\Gi{e}$ & overall rate of $e$ for individual~$i$ \\
  $\Ri{e}$ & base rate of $e$ for individual~$i$ \\
  $\RR{e}{x}$ & rate ratio of $e$ given $x$ \\
  $tRR$    & transient rate ratio\tn{1} \\
  $dRR$    & duration rate ratio\tn{1} \\
  $\tau$   & time scale of $tRR$ and $dRR$\tn{1} \\
  $iRR$    & initial value of $tRR$\tn{1} \\
  $nRR$    & cumulative rate ratio\tn{1} \\
  $\eta$   & count scale of $nRR$\tn{1} \\
  $mRR$    & maximum / minimum value of $nRR$\tn{1} \\
  \bottomrule
\end{tabular}
\floatfoot{\centering
  \tnt[1]{See \sref{mod.par.evt.vef} for details.}}

\end{table}
%------------------------------------------------------------------------------
\paragraph{Implementation}
The model uses a fixed timestep of $\dt = 7$ days,%
\footnote{One year is thus defined as $52\dt$.}
reflecting a trade-off between temporal resolution and computational efficiency.
Event-scheduling approaches were considered infeasible because
multiple ``upstream'' events are modelled to have
immediate effects on the rates of ``downstream'' events
(\eg exposure to SGM violence
immediately increases the rate of depression onset);
thus, ``downstream'' events cannot be scheduled.
Instead, each timestep, the probability of each event occurring
is computed for all individuals aged 10--59, using vectorized code.
The model is implemented in R (v3.6.3), with code available online.%
\footnote{\hreftt{github.com/jessexknight/shift}}
Table~\ref{tab:notation} provides a notation reference.
