%%%%%%%%%%%%%%%%%%%%%%%%%%%%%%%%%%%%%%%%%%%%%%%%%%%%%%%%%%%%%%%%%%%%%%%%%%%%%%%
\section{Simulation Model Parameterization}\label{mod.par}
Table~\ref{tab:par} summarizes the model parameters,
while the details of the modelled processes are given below.
\begin{table}
  \vskip-2\baselineskip
  \caption{\note{Overview of model parameters}
    {Many of the RR "values" include several sub-strata (e.g. ages)
     or reflect non-linear functions (e.g. transient and cumulative effects),
     so I've opted to provide links to Tables with more details these RR,
     e.g. underlying parameters  example RR.
     Also Might need to split this up or split across pages
     once the actual values (and ranges) for both priors \& posteriors are added.}}
  \label{tab:par}
  \centering\begin{tabular}{llll}
  \toprule
  Variable & Stratification & \clap{Symbol\tn{1}} & Value\tn{2} \\
  % MCB: maybe use same definition as in table A.1 and also Relative rate or rate ratio etc
  \midrule
  Active population size           &             & $N$     &  10,000 \\
  % MCB: i would specific sexually active
  Active population ages           &             & ---     & [10,~60) \\
  Age of starting sexual activity  &             & ---     & Table~\ref{tab:age.act} \\
  Background rate of mortality     & for age $a$ & \ss Rma & Table~\ref{tab:rr.age} \\
  \midrule
  Base rate of violence exposure             & *                                   & \Ri{v}       & ?? \\
  % MCB: what is this supposed to represent - base rate for overall pop
  %      in absence of age effect or base rate for younger ones or?
  Relative rate of violence exposure         & for age $a$ (years)                 & \RR{v}{a}    & Table~\ref{tab:rr.age} \\
  \midrule
  Base rate of depression onset              & *                                   & \Ri{d}       & ?? \\
  Relative rates of depression onset         & for age $a$ (years)                 & \RR{d}{a}    & Table~\ref{tab:rr.age} \\
                                             & if any previous depression          & \RR{d}{d'}   & ?? \\
                                             & for violence exposure \du days ago  &\tRR{d}{v}    & Table~\ref{tab:xrr} \\
                                             & per violence exposure (ever)        &\nRR{d}{v}    & Table~\ref{tab:xrr} \\
  % MCB: per number of exposures? if u=lifetime above does that mean that we are multiplying the RR twice if we also specify RR>1 here?
  %      also, what does this (n) represent?
  Base rate of depression recovery           & *                                   & \Ri{\d}      & ?? \\
  Relative rate of depression recovery       & for episode duration \du days       &\dRR{\d}{\du} & Table~\ref{tab:xrr} \\
                                             & for violence exposure \du days ago  &\tRR{\d}{v}   & Table~\ref{tab:xrr} \\
  % MCB: maybe use the term onset here and above where relevant
  \midrule
  Base rate of \hazdrink onset               & *                                   & \Ri{h}       & ?? \\
  Relative rates of \hazdrink onset          & for age $a$ (years)                 & \RR{h}{a}    & Table~\ref{tab:rr.age} \\
                                             & if any previous \hazdrink           & \RR{h}{h'}   & ?? \\
                                             & for violence exposure \du days ago  & \tRR{h}{v}   & Table~\ref{tab:xrr} \\
                                             & per violence exposure (ever)        &\nRR{h}{v}    & Table~\ref{tab:xrr} \\
                                             & if current depression               & \RR{h}{d}    & ?? \\
  Base rate of \hazdrink recovery            & *                                   & \Ri{\h}      & ?? \\
  Relative rate of \hazdrink recovery        & for episode duration \du days       &\dRR{\h}{\du} & Table~\ref{tab:xrr} \\
                                             & for violence exposure \du days ago  &\tRR{\h}{v}   & Table~\ref{tab:xrr} \\
                                             & if current depression               & \RR{\h}{d}   & ?? \\
  % MCB: "currently depressed"
  \midrule
  Maximum number of concurrent partners      & *                                   & \Mi{p}       & \sref{mod.par.evt.ptr} \\
  Base rate of partnership formation         & *                                   & \Ri{p}       & ?? \\
  Relative rate of partnership formation     & for age $a$ (years)                 & \RR{p}{a}    & Table~\ref{tab:rr.age} \\
  % MCB: we need to be relative to what here and elsewhere to avoid ambiguity
                                             & for violence exposure \du days ago  &\tRR{p}{v}    & Table~\ref{tab:xrr} \\
                                             & per violence exposure (ever)        &\nRR{p}{v}    & Table~\ref{tab:xrr} \\
                                             & if current depression               & \RR{p}{d}    & ?? \\
                                             & if current \hazdrink                & \RR{p}{h}    & ?? \\
  Base rate of partnership dissolution       & *                                   & \Ri{\p}      & ?? \\
  Relative rate of partnership dissolution   & for age $a$ (years)                 & \RR{\p}{a}   & Table~\ref{tab:rr.age} \\
                                             & for violence exposure \du days ago  &\tRR{\p}{v}   & Table~\ref{tab:xrr} \\
                                             & if current depression               & \RR{\p}{d}   & ?? \\
                                             & if current \hazdrink                & \RR{\p}{h}   & ?? \\
  \midrule
  Base probability of condom use per sex act & *                                   & \Pi{c}       & ?? \\
  Relative probability of condom use         & for age $a$ (years)                 & \RP{c}{a}    & Table~\ref{tab:rr.age} \\
  % MCB: per sex act? should be - or could also be non  condom use if condom use is high
                                             & for violence exposure \du days ago  &\tRP{c}{v}    & Table~\ref{tab:xrr} \\
                                             & per violence exposure (ever)        &\nRP{c}{v}    & Table~\ref{tab:xrr} \\
                                             & if current depression               & \RP{c}{d}    & ?? \\
                                             & if current \hazdrink                & \RP{c}{h}    & ?? \\
  \bottomrule
\end{tabular}
\floatfoot{
  \tnt[1]{}%
  \Ri{}{}: individual-level base rate;
  \RR{}{}: population-level relative rate;
  \tRR{}{} / \nRR{}{}: transient / cumulative relative rates
    (see \sref{mod.par.evt.vef} for details);
  % MCB: might be better to include in the table
  \tnt[2]{all durations in days; all rates in per-day}
  % MCB: per capita
}
% MCB: also needs an idea of all these combined

\end{table}
%==============================================================================
\subsection{Active Population}\label{mod.par.act}
The active population is defined as SGM individuals aged 10--59,
% MCB: do you mean sexually active? & is it what we want to model ?
which aims to capture the majority of sexual activity among SGM \cite{??},
plus exposure to SGM violence, depression, and \hazdrink,
including possibly prior to sexual activity.
% MCB: i dont understand what you mean here -
%      above MSM enters unexposed to everything and you model sexually active
%      so how can experience before sexual activity be represented?
Individuals become sexually active at a randomly sampled age
\note{from the distribution}
     {This could come from data -- at least loosely.
      It is currently set to a linear ramp from 0-100\% between ages 10 and 20.}
given in Table~\ref{tab:age.act}.
Individuals exit the active population by reaching age 60,
\note{or through background mortality before age 60.}
     {This is not currently implemented
      and we may want to discuss whether it is needed in the validation study.
      Well, I will implement it and see how it affects any results.}
Mortality rates are given in Table~\ref{tab:rr.age},
loosely reflecting ??
\begin{table}
  \caption{Ages of first sexual activity}
  \label{tab:age.act}
  \centering\input{tab.age.act}
\end{table}
%==============================================================================
\subsection{Event Rates}\label{mod.par.evt}
Event rates are modelled as a product of
randomly sampled \emph{individual-specific} base rates \Ri{} and
% MCB: these should be defined per year and rescaled in the code
fixed \emph{population-level} relative rates $RR$
given an individual's current age,
prior SGM violence exposure(s), depression status, and \hazdrink status.
The resulting rate of event~$e$ for individual~$i$ is denoted $\Gi{e}$.
% MCB: i think you called is overall rate above
\par
Some events are assumed to occur up to once per timestep per individual:
% MCB: do you mean a maximum of once pet time step? Can more than one event happens per time step?
depression onset, depression recovery,
\hazdrink onset, and \hazdrink recovery.
% MCB: why?
For these events, the probability of the event occurring to individual~$i$
during the timestep $\dt$ is defined as:
\begin{equation}\label{eq:r2p}
  \Pi{e} = 1 - \exp(-\Gi{e}\dt)
\end{equation}
Other events are modelled to occur
any number of times per timestep per individual:
exposure to SGM violence,
sexual partnership formation,
% MCB: why? because of partnerships with only one contact of with FSW?
%      what about dissolution ? make sure you have really described everything
and sex within partnerships.
% acts - do you have a symbol for each event? need to
For these events, the number of events occurring to individual~$i$
during the timestep $\dt$ is:%
\footnote{\eqref{eq:r2p} simply reflects
  the probability of $\Ni{e} > 0$ in \eqref{eq:r2n}.}
\begin{equation}\label{eq:r2n}
  \Ni{e} \sim \pois(\Gi{e}\dt)
\end{equation}
%------------------------------------------------------------------------------
\begin{table}
  \caption{Rates / relative rates of events by age}
  \label{tab:rr.age}
  \centering\input{tab.rr.age}
\end{table}
\begin{figure}[t]
  \centering[TODO]
  \caption{Relative rates of events by age}
  \label{fig:age.rr}
\end{figure}
%------------------------------------------------------------------------------
\subsubsection{Exposure to SGM Violence}\label{mod.par.evt.vex}
Individuals are modelled to have
a unique base rate of exposure to SGM violence (\Ri{v}) throughout their life,
% MCB: do you mean all individuals are assumed to have the same unique base rate?
reflecting differences in individuals, family, friends, and the broader local community
\cite{??}.
This base rate is drawn from a population-level distribution (Figure~\ref{??}).
\note{A calibrated relative rate by age}
     {I have set this to 1 for now.}
\RR{v}{a} is also applied equally to all individuals.
% TODO: read more papers to inform this
% As shown in Figure~\ref{fig:age.rr},
% this relative rate:
% increases during ages
% 10--20,
% possibly reflecting increasing vulnerability while coming out
% and identifying safe family, friends, and locations;
% and later decreases during ages 30--60 \cite{Muller2021},
% possibly reflecting reduced sexual activity and/or
% accumulation of knowledge and behaviours
% about how to avoid SGM violence.
Thus, the rate of
exposure to SGM violence for individual~$i$ is modelled~as:
\begin{equation}
  \Gi{v} = \Ri{v} \cdot \RR{v}{a}(a_i)
\end{equation}
where $a_i$ is the age of individual~$i$.
An individual's history of exposure to SGM violence
% MCB: lifecourse
is then represented by the set of times in which they were exposed: $t^v_i$,
% MCB: when they were exposed- is that a single event or a period of time?
which can include duplicate values,
% MCB: not clear what this means
if the individual was exposed more than once in any timestep
--- \ie \eqref{eq:r2n} $> 1$.
% MCB: you could have a little diagram over time of what it can look like
%      i.e. just an arrow and representing the violence events -
%      and you can carry this representation to the next sections
%      to add the different events so we can see how this may look like for an individual i
%------------------------------------------------------------------------------
\subsubsection{Effects of SGM Violence}\label{mod.par.evt.vef}
Exposure to SGM violence is modelled to have two types of effects,
% MCB: assumed instead of modelled?
both applied as selected relative rates of downstream events:
a transient effect ($\tRR{}{}(t^v_i)$, where
$t^v_i$ is the set of times of recent exposure to SGM violence for individual~$i$), and
a cumulative effect ($\nRR{}{}(n^v_i)$, where
$n^v_i$ is the total number of lifetime exposures to SGM violence for individual~$i$).
%------------------------------------------------------------------------------
\paragraph{Transient Effects}
Transient effects of SGM violence are applied as relative rates of:
% MCB: maybe would be easier to visualise if you listed these a bullet points below
depression onset (\tRR{d}{v}),
depression recovery (\tRR{\d}{v}),
\hazdrink onset (\tRR{h}{v}),
\hazdrink recovery (\tRR{\h}{v}),
partnership formation (\tRR{p}{v}), and
partnership dissolution (\tRR{\p}{v}), plus
\note{a relative \emph{probability} of condom use}
     {I have not implemented this yet either.}
% MCB: per sex act? and could be per non condom use
% MCB: i dont think you have explained you notation
%      for example what do you use the supercript to denote, subscript,etc
%      - not the specific but the general notation
(\tRP{c}{v}).%
\footnote{To ensure this probability remains bounded $\in [0,1]$,
  we model either $RP \le 1$ of condom use, or $RP \le 1$ of condom non-use.}
% MCB: this need to be clear which one is used
These relative rates are modelled to wane
\note{exponentially towards $RR = 1$}
     {As you know, we can have other shapes,
      and I am actually leaning towards linear (ramp)
      for the first paper since it is a nice compromise between
      having precise properties (area under the curve, time to RR=1)
      while still seeming "visually plausible".}
% MCB: here you could write "are modelled as a function
%      that has the ability to reflect waning of effect over time"
%      but default s value is ... whatever you are representing.
%      it will be very important that it is clear what the model can do
%      and what you actually assumed in the simulations, which can be different
(\ie no effect) with
\note{time since the most recent exposure}
     {Originally, I had an additive model,
      where two exposures in the same timestep would effectively double the tRR.
      However, in this new version, the most recent exposure
      will simply "reset" the tRR to its maximum (iRR),
      so there will be no difference between 1 or 2+ exposures in the same timestep.
      Honestly, this was motivated by code speed-up
      which only works with the new approach, but I can revert if needed.
      It may be hard to find data to support either approach.}
$\du^v_i = t - t^v_i$.
% MCB: did you define both before? and in particular t?
The transient effects of the most recent exposure is thus specified as follows:
\begin{equation}\label{eq:tRR}
  \tRR{}{}(\du^v_i) = 1+(\iRR{}{}-1)\exp(-\du^v_i/\tau)
\end{equation}
where:
\iRR{}{} is an initial relative rate, and
% MCB: is the notation correct - is IRR specific to individual i?
$\tau$ is a scale parameter controlling the duration of effect.%
\footnote{For example, \tRR{}{} decays to $\approx 0.37~\iRR{}{}$ by $t = \tau$.}
\par
When considering discrete timesteps (indexed $z$, with length \dt),
% MCB: what is this number of time steps?
we replace $t$ above with $z\,\dt$ and $t^v_i$ with $z^v_i$.
Transient effects of violence are further assumed to
\note{only begin during the timestep following exposure,}
     {This part is quite tricky when considering
      *when* within a timestep events occur ---
      and thus when the effect is at it's maximum
      and by how much it has decayed *during* the next timestep.
      I have gone back-and-forth with offsets of 1/2 timestep,
      so it might be good to discuss (with Mike \& with pictures!)
      --- Edit: I now think we probably need some kind of integration
      over the continuous-time definition of tRR for each subsequent timestep,
      (with violence exposure defined as the mid-point
      and thus with a half-effect within the *same* timestep).}
and thus only define \tRR{}{} for $\du^v_i > 0$.
Figure~\ref{fig:tRR} illustrates
an example transient effect $\tRR{}{}(\du^v_i)$ given by \eqref{eq:tRR}.
\begin{figure}
  \centering
  \begin{subfigure}{.5\linewidth}
    \includegraphics[width=\linewidth]{tRR.exp}
    \caption{Transient effect ($tRR$) \vs timesteps since exposure}
    % MCB: (iRR) why are all tRR below this value- should it not start at this?
    %      Also, add symbols in brackets (x axis label)
    \label{fig:tRR}
  \end{subfigure}%
  \begin{subfigure}{.5\linewidth}
    \includegraphics[width=\linewidth]{nRR.exp}
    \caption{Cumulative effect ($nRR$) \vs number of exposures}
    \label{fig:nRR}
  \end{subfigure}
  \begin{subfigure}{.5\linewidth}
    \includegraphics[width=\linewidth]{dRR.exp}
    \caption{Duration effect ($nRR$) \vs duration since onset}
    \label{fig:dRR}
  \end{subfigure}
  \caption{Illustration of transient, cumulative, and duration effects as relative rates}
  % MCB: might need more explanation in figure legend in words
  \label{fig:xRR}
\end{figure}
%------------------------------------------------------------------------------
\paragraph{Cumulative Effects}
% MCB: edit as suggested above
Cumulative effects of SGM violence are applied as relative rates of
depression onset (\nRR{d}{v}),
\hazdrink onset (\nRR{h}{v}), and
partnership formation (\nRR{p}{v}), plus
a relative \emph{probability} of condom use (\nRP{c}{v}).
These relative rates are modelled to accumulate exponentially
with each additional exposure, up to a maximum value:
\begin{equation}\label{eq:nRR}
  \nRR{}{}(n^v_i) = 1+(\mRR{}{}-1)(1-\exp(-n^v_i/\eta))
\end{equation}
% MCB: again here you need to say we can do this
%      but also make the simplext asumption of constant effect for x time steps
where $n^v_i$ reflects the cumulative number of
exposures to SGM violence for individual~$i$.
% MCB: is their a recall period over which we cumulate experiences?
Cumulative effects are thus also specified via 2 parameters:
a maximum relative rate \mRR{}{}, and a scale parameter $\eta$.%
\footnote{Specifically, $\nRR{}{} \approx 1 + 0.63 (\mRR{}{}-1)$
  after $n = \eta$ cumulative exposures.}
% MCB: is that (eta) defined?
Figure~\ref{fig:nRR} illustrates the general shape of \nRR{}{}
with an increasing number of cumulative exposures.
Like transient effects, cumulative effects are also assumed
to begin the next timestep following exposure.
% MCB: so essentially instantaneously  - could that by varied?
%      for eg could we use function a) to have an increasing function instead?
\begin{table}
  \caption{Summary of modelled non-linear effects (relative rates) due to
    exposure to SGM violence and
    duration of depression/\hazdrink episodes}
  \label{tab:xrr}
  \centering\newcommand{\unit}[1]{\quad$\clap{#1:}$}
\newcommand{\xmidrule}{\cmidrule(rl){1-1}\cmidrule(rl){2-3}\cmidrule(rl){5-9}}
\begin{tabular}{lccrrrrrr}
  \toprule
  Affected Rate\tn{1} & \multicolumn{2}{c}{Parameter values\tn{2}} & &\multicolumn{5}{c}{Example \RR{}{}} \\
  \midrule
  Transient effects of violence
  & Initial RR (\iRR{}{}) & Time scale ($\tau$) & \unit{\du} & 1 &  3 & 10 & 30 & 100 \\
  % MCB: (tau) not sure what is presented below simply time step or number of steps
  %      and i think that parameters should be given in real time scale not time steps
  \xmidrule
  \tRR{d}{v}:  depression onset        & ?? & ?? & & ?? & ?? & ?? & ?? & ?? \\
  \tRR{\d}{v}: depression recovery     & ?? & ?? & & ?? & ?? & ?? & ?? & ?? \\
  \tRR{h}{v}:  \hazdrink onset         & ?? & ?? & & ?? & ?? & ?? & ?? & ?? \\
  \tRR{\h}{v}: \hazdrink recovery      & ?? & ?? & & ?? & ?? & ?? & ?? & ?? \\
  \tRR{p}{v}:  partnership formation   & ?? & ?? & & ?? & ?? & ?? & ?? & ?? \\
  \tRR{\p}{v}: partnership dissolution & ?? & ?? & & ?? & ?? & ?? & ?? & ?? \\
  \tRP{c}{v}:  condom use              & ?? & ?? & & ?? & ?? & ?? & ?? & ?? \\
  \midrule
  Cumulative effects of violence
  & Max RR (\mRR{}{}) & Count scale ($\eta$) & \unit{$n$} & 1 &  3 & 10 & 30 & 100 \\
  \xmidrule
  \nRR{d}{v}: depression onset       & ?? & ?? & & ?? & ?? & ?? & ?? & ?? \\
  \nRR{h}{v}: \hazdrink onset        & ?? & ?? & & ?? & ?? & ?? & ?? & ?? \\
  \nRR{p}{v}: partnership formation  & ?? & ?? & & ?? & ?? & ?? & ?? & ?? \\
  \nRP{c}{v}: condom use             & ?? & ?? & & ?? & ?? & ?? & ?? & ?? \\
  \midrule
  Effects of episode duration
  &  & Time scale ($\tau$) & \unit{\du} & 1 & 3 & 10 & 30 & 100 \\
  \xmidrule
  \dRR{\du}{d}: depression recovery &   & ?? & & ?? & ?? & ?? & ?? & ?? \\
  \dRR{\du}{h}: \hazdrink recovery  &   & ?? & & ?? & ?? & ?? & ?? & ?? \\
  \bottomrule
\end{tabular}
\floatfoot{
  \tnt[1]{All durations in days; all rates in per-day.}
  % MCB: we prefer per year as more readily interpretable
  \tnt[2]{See Eqs.~(\ref{eq:tRR}--\ref{eq:nRRv}), (\ref{eq:nRRd}) for parameter details.}
}

\end{table}
% TODO: depression tRR -> dRR (fundamentally different)
%------------------------------------------------------------------------------
\subsubsection{Depression}\label{mod.par.evt.dep}
Although depression is likely a complex and dynamic system \cite{Cramer2016},
for simplicity, we conceptualize depression as binary,
using a working definition of
\note{PHQ-9 score $\ge 10$}
     {We may need to review whether PHQ-9 has been validated on younger ages
      --- the Kroenke2001 paper had ages 18+}
\cite{Kroenke2001}.
Depression status is denoted:
$d_i = 1$ if individual~$i$ is currently depressed and $d_i = 0$ otherwise; plus
$d'_i = 1$ if individual~$i$ was ever previously depressed and $d'_i = 0$ otherwise.
% MCB: this goes after the following sentence- you first model depression onset and recovery
%      and depression is based on such and such definition
%      & we keep track of  with the following indicators
Each individual is initialized with
a base rate of depression onset (\Ri{d}) and
a base rate of depression recovery (\Ri{\d}),
reflecting differences in susceptibility to depression \cite{Hankin2015}.
%------------------------------------------------------------------------------
\paragraph{Depression Onset}
Individuals are modelled to be at risk of depression onset
from age 10 and not before~\cite{Solmi2022}.
The rate of depression onset is modelled to
% MCB: base rate?
\note{increase over ages 10--20 and decline after age 30}
     {As before, all age RRs are currently set to 1.}
(\RR{d}{a}, Table~\ref{tab:rr.age}, Figure~\ref{fig:age.rr})
\cite{Patten2010,Hankin2015,Solmi2022}.
The rate of depression onset is also modelled to increase
% MCB: to be increased by a multiplicative factor  with ...
by \RR{d}{d'}
\note{with any previous depression}
     {It may be hard to untangle this effect from
      individual-level heterogeneity (Ri) in susceptibility to onset,
      as they would both manifest as
      the rate of onset (relapse) among previously depressed being greater than
      the rate of onset among those never depressed.}
\cite{Kendler2010},
and increase with exposure to SGM violence
via transient (\tRR{d}{v}) and cumulative (\nRR{d}{v}) effects,
as described in \sref{mod.par.evt.vef}.
Thus, the rate of depression onset for individual~$i$
(who is not currently depressed) is given by
\begin{equation}
  \Gi{d} = \Ri{d}
     \cdot \RR{d}{a}(a_i)
     \cdot \RR{d}{d'}(d'_i)
     % MCB: what is this (d') again?
     \cdot \tRR{d}{v}(t^v_i)
     \cdot \nRR{d}{v}(n^v_i)
\end{equation}
In a small abuse of notation, let
\note{$RR(x)$ denote $1+x{\cdot}(RR-1)$}
     {Not sure how you feel about this notation?
      Spelling out the math fully in each equation looks very messy.}
when $x \in \{0,1\}$
--- \ie $\{RR\txs{if}x=1\txs{else}1\}$.
\paragraph{Depression Recovery}
Individuals may recover from depression ($\d$)
in the timestep immediately after onset,
although the minimum duration of symptoms
in clinical criteria for depression is 2 weeks \cite{APA2013}.
The relative rate of depression recovery is modelled to decline
with longer and longer duration depressed during this episode $\du^d_i$,
% MCB: maybe replace by increasingly longer xxx
as suggested by prior empirical modelling \cite{Patten2005}
% MCB: hum - not sure what this means - is it empirical data or modelling?
\note{.}{When calibrating to data, we should consider another Patten paper:
         https://doi.org/10.1186/1471-244X-10-85
         which suggests some people fail to recall past depression episodes
         --- though I don't think our 3 longitudinal datasets
         ask about *past* depression anyways.}
This effect is thus modelled to wane exponentially from 1 to 0:
\begin{equation}\label{eq:dRR}
  \dRR{}{}(\du^d_i) = \exp(-\du^d_i/\tau)
\end{equation}
where: $\tau$ is a scale parameter as before.
Figure~\ref{fig:dRR} illustrates
an example transient effect $\dRR{}{}(\du^d_i)$ given by \eqref{eq:dRR}.
The rate of depression recovery is also modelled to
transiently decrease with exposure to SGM violence (\tRR{\d}{v}),
as described in \sref{mod.par.evt.vef}.
Effects of age on recovery are not considered.
% MCB: "We assum that age does not influence xxx"
Thus, the rate of depression recovery for depressed individual~$i$ is given by:
\begin{equation}
  \Gi{\d} = \Ri{\d}
    \cdot \dRR{\d}{\du}(\du^d_i)
    \cdot \tRR{\d}{v}(t^v_i)
\end{equation}
%------------------------------------------------------------------------------
\subsubsection{Hazardous Drinking}\label{mod.par.evt.haz}
% MCB: see similar suggestions as depression for order of info presented
Hazardous drinking is also conceptualized as binary,
using the working definition of AUDIT-C score $\ge 4$ \cite{Bush1998}.
It is modelled almost identically to depression
(with different parameter values),
except that effects of current depression status ($d$)
on \hazdrink onset and recovery are also considered,
per the hypothesized causal pathway (Figure~\ref{fig:dag}).
%------------------------------------------------------------------------------
\paragraph{Hazardous Drinking Onset}
Thus, the rate of \hazdrink onset for individual~$i$
is modelled as the product of
an individual-specific base rate \Ri{h},
relative rates for age (\RR{h}{a}),
current depression (\RR{h}{d}),
and any previous \hazdrink (\RR{h}{h'}), plus
transient and cumulative relative rates for exposures to SGM violence
(\tRR{h}{v}, \nRR{h}{v})
\note{:}
     {As Mike noted, having many RR could result in implausibly high total rates.
      So, we can revisit some ways to constrain it later.}
\begin{equation}
  \Gi{h} = \Ri{h}
     \cdot \RR{h}{a}(a_i)
     \cdot \RR{h}{h'}(h'_i)
     \cdot \RR{h}{d}(d_i)
     \cdot \tRR{h}{v}(t^v_i)
     \cdot \nRR{h}{v}(n^v_i)
     % MCB: for all these risk factors we will want flexibility
     %      on how these are modelled because they may end up being quite high.
\end{equation}
%------------------------------------------------------------------------------
\paragraph{Hazardous Drinking Recovery}
Likewise, the rate of \hazdrink recovery for individual~$i$
is modelled as the product of
an individual-specific base rate \Ri{\h},
relative rates for duration drinking hazardously (\dRR{\h}{\du}),
and current depression (\RR{\h}{d}), plus
a transient relative rate for exposure to SGM violence (\tRR{\h}{v}):
\begin{equation}
  \Gi{\h} = \Ri{\h}
    \cdot \RR{\h}{d}(d_i)
    \cdot \dRR{\h}{\du}(\du^h_i)
    \cdot \tRR{\h}{v}(t^v_i)
\end{equation}
%------------------------------------------------------------------------------
\subsubsection{Sexual Partnerships}\label{mod.par.evt.ptr}
Only sexual partnerships formed within the modelled SGM population are considered.
% MCB: wording perhaps a bit strange
%      Perhaps re write saying the the individuals can only form partnerships
%      within the simulated population and not with individual outside of the population.
That is, no ``external'' or heterosexual partnerships are modelled,
and all sexual partnerships formed by individuals in the modelled population
are assumed to be formed with other individuals in the modelled population.
%------------------------------------------------------------------------------
\paragraph{Sexual Partnership Formation}
Each individual is modelled to have two unchanging characteristics
% MCB: for life?
that influence their rate of sexual partnership formation:
a maximum number of concurrent partners \Mi{p}, and
a base rate of sexual partnership formation \Ri{p}
(while having fewer than \Mi{p} partners).
% MCB: that is bounded by the max number of concurrent partners that they can have
\par
Relative rates of sexual partnership formation are also implemented for
individuals' age (\RR{p}{a}),
exposure to SGM violence (\tRR{p}{v}, \nRR{p}{v}),
current depression status (\RR{p}{d}), and
current \hazdrink status (\RR{p}{h}).
Thus, the rate of sexual partnership formation for individual~$i$
(who has fewer than \Mi{p} current partners) is given by:
\begin{equation}
  \Gi{p} = \Ri{p}
    \cdot \RR{p}{a}(a_i)
    \cdot \tRR{p}{v}(t^v_i)
    \cdot \nRR{p}{v}(n^v_i)
    \cdot \RR{p}{d}(d_i)
    \cdot \RR{p}{h}(h_i)
\end{equation}
Individuals may form multiple new sexual partnership per timestep via \eqref{eq:r2n}.
%------------------------------------------------------------------------------
\paragraph{Sexual Partnership Dissolution}
Sexual partnerships are modelled to dissolve at a rate
influenced equally by the characteristics of both partners (denoted $i,j$\,).
A base rate is modelled as the mean of individual-specific base rates:
$\frac12(\ss R\p i + \ss R\p j)$.
This base rate is then multiplied by relative rates for
\note{\emph{each} partner's}
     {e.g. if both partners are depressed, and RR = 2, then we currently have RR total = 4;
      We could alternatively chose the maximum of effects (of each variable) among the 2 partners,
      or something similar to this
      --- I think this came up one meeting with Mike.}
age (\RR{\p}{a}),
recent exposure to SGM violence (\tRR{\p}{v}),
current depression status (\RR{\p}{d}), and
current \hazdrink status (\RR{\p}{h}),
to obtain the overall rate:
\begin{align}
  \Gi{\p} = \textstyle\frac12 (\Ri{\p} + \ss R\p j)
   &\cdot \RR{\p}{a}(a_i)
    \cdot \tRR{\p}{v}(t^v_i)
    \cdot \RR{\p}{d}(d_i)
    \cdot \RR{\p}{h}(h_i)
    \nonumber \\
   &\cdot \RR{\p}{a}(a_j)
    \cdot \tRR{\p}{v}(z^v_j)
    \cdot \RR{\p}{d}(d_j)
    \cdot \RR{\p}{h}(h_j)
\end{align}
\note{All partnerships are modelled to last at least 1 timestep (7 days).}
     {Well, this assumption may need to be revised based on the validation findings
      (results in underestimation of dissolution rate).}
%------------------------------------------------------------------------------
\subsubsection{Anal Sex \& Condom Use}
[TODO]
% \max(\ss Pci, \ss Pcj)
%------------------------------------------------------------------------------
\subsubsection{Correlated Parameters}
% MCB: ?
\note{[TODO]}
     {I implemented crude correlations among individual-level base rates to capture the fact that
      individuals who are susceptible to depression onset are also less likely to recover,
      and likewise for hazdrink; similarly for partnerships:
      individuals with higher maximum number of concurrent partners
      will have higher formation + dissolution rates.
      However since we have been focusing on homogeneous populations for validation,
      this currently has no effect.}
% These characteristics are modelled to be correlated,
% reflecting empirically observed correlations between
% sexual partnership concurrency and numbers of lifetime sexual partners \cite{Warren2015};
% this correlation is implemented using a Gaussian copula \cite{Sklar1959}
% with correlation coefficient $\rho$.
%------------------------------------------------------------------------------
\subsubsection{Summary}\label{mod.par.evt.sum}
In total, 29 relative rates (RR) or relative probabilities (RP) are considered
in the simulation model (Table~\ref{tab:par}, Figure~\ref{fig:dag}), including:
\begin{itemize}
  \item 6 RR associated with individuals' ages
  \item 11 RR due to SGM violence exposure (%
    7 transient and 4 cumulative)
  \item 7 RR due to depression status/history (%
    1 for any previous depression,
    1 for current depression episode duration, and
    5 for current depression status)
  \item 5 RR due to \hazdrink status/history (%
    1 for any previous \hazdrink,
    1 for current \hazdrink episode duration, and
    3 for current \hazdrink status).
\end{itemize}
%==============================================================================
\subsection{Sexual Mixing}\label{mod.par.mix}
Sexual mixing refers to formation of sexual partnerships
according to individuals' characteristics.
Among individuals who are determined to form new sexual partnerships
within a given timestep (see \sref{mod.par.evt.ptr}),
sexual mixing is currently assumed to be fully random.
% MCB: or do you mean proportionate? - random is a bit ambiguous.
%      I think you need to say how you define the mixing -
%      i dont have a good sens of how it is specificed in the model-
%      how the mixing probability are derived as should be based on
%      number of partners and number of people and preference matrix
